\documentclass[a4paper,12pt]{article}
\usepackage[utf8]{inputenc}
\usepackage[russian]{babel}
\usepackage{geometry}
    \geometry{left=2cm,right=2cm,top=2cm,bottom=2cm}
\usepackage{amsmath}
\usepackage{indentfirst}
\usepackage{hyperref}
\usepackage{fancyhdr}

% Устанавливая стиль страницы без нумерации на титульной странице
\pagestyle{empty}

% Создавая титульный лист
\title{}
\author{}
\date{}

\begin{document}

\begin{center}
    \textbf{МИНИСТЕРСТВО НАУКИ И ВЫСШЕГО ОБРАЗОВАНИЯ РОССИЙСКОЙ ФЕДЕРАЦИИ}

    \vspace{0.1cm}

    Федеральное государственное автономное образовательное учреждение высшего образования

    \vspace{0.1cm}

    \textbf{НАЦИОНАЛЬНЫЙ ИССЛЕДОВАТЕЛЬСКИЙ \\ТОМСКИЙ ПОЛИТЕХНИЧЕСКИЙ УНИВЕРСИТЕТ}

    \vspace{1.0cm}

    Инженерная школа информационных технологий и робототехники

    Отделение информационных технологий

    Направление информатика и вычислительная техника

    \vspace{2cm}

    \textbf{Отчет} \\
    по лабораторной работе №1 \\
    \vspace{0.5cm}
    по дисциплине \\
    «\MakeUppercase{Прога}» \\
    \vspace{0.5cm}
    \textbf{Отчёты}
\end{center}

\vspace{1.5cm}

\begin{flushleft}
    Выполнил: \\
    Студент группы Тест
\end{flushleft}
\begin{center}
    \underline{\hspace{5cm}}
\end{center}
\begin{flushright}
    Тест
\end{flushright}

\vspace{0.5cm}

\begin{flushleft}
    Проверил: \\
    Ассистент ОИТ ИШИТР
\end{flushleft}
\begin{center}
    \underline{\hspace{5cm}}
\end{center}
\begin{flushright}
    А.Ю. Малкин
\end{flushright}

\vspace{4.4cm}

\begin{center}
    Томск 2025
\end{center}

% Включая нумерацию страниц начиная со второй страницы
\newpage
\pagestyle{fancy}
\fancyhf{}
\fancyfoot[R]{\thepage}

% Описывая содержание разделов
\section*{Цель работы}
Тест1

\section*{Задание}
Тест2

\section*{Ход работы}
Тест3

\newpage

\section*{Выводы}
Тест1

\newpage

\begin{flushright}
\textbf{Приложение 1 - Код программы}
\end{flushright}
По ссылке, ведущий на репозиторий в codelab.tpu.ru, представлены файлы исходного кода проекта к данной лабораторной работе: \href{repo}{repo}

\end{document}